%! TEX root = ../rulebook.tex

\nonumsubsection{Park Factors}{sec:parkfactors}

After a total range is calculated, park factors may be applied to it if the game is utilizing them 
(that is, if the home team has created a park with custom park factors).

Park factors work as a modifier similar to the pitcher hand bonus, 
but is applied as multipliers for specific plate appearance outcomes. 
A typical park may look like this:

\begin{center}
    \begin{tabular}{|c|c|c|c|c|c|}
        \hline 
        & HR & 3B & 2B & 1B & BB \\
        \hline 
        No Stadium & 1.000 & 1.000 & 1.000 & 1.000 & 1.000 \\
        \hline 
        Fake Shea Stadium & 0.879 & 1.042 & 1.032 & 0.963 & 1.000 \\
        \hline
    \end{tabular}
\end{center}

These multipliers act as modifiers on those specific outcomes. 
With park factor adjustment, an outcome's size is thus multiplied by the given factor 
and rounded to the nearest integer to get an adjusted size.

\begin{center}
    \begin{tabular}{|c|c|c|c|c|c|}
        \hline 
        & HR & 3B & 2B & 1B & BB \\
        \hline 
        Combined Range for AB & 20 & 4 & 29 & 65 & 39 \\
        \hline 
        Fake Shea Stadium & 0.879 & 1.042 & 1.032 & 0.963 & 1.000 \\
        \hline
        Modified Range & 17.58 & 4.168 & 29.928 & 62.595 & 39 \\
        \hline
        \textbf{Rounded Final Range} & 18 & 4 & 30 & 63 & 39 \\
        \hline 
    \end{tabular}
\end{center}

In this case, a total of 3 points are taken away from hitter-beneficial ranges. 
To balance this and keep the total range size at 501, the points are equally distributed to each pitcher-beneficial range.

In the case of a net addition to hitter-beneficial ranges, points are taken out of pitcher-beneficial ranges, also contributed equally. 
In some cases, a range may be too small to offer any additional points to balance the scale. 
In this case, the range contributes all of its points and remains at size 0. 
The remaining needed points are contributed from the other ranges.

\textbf{Creation Rules:}\\
The next table represents the Minimum and Maximum modifiers that can be given to each outcome during park creation.

\begin{center}
      \begin{tabular}{|c|c|c|c|c|c|}
	\hline
	& HR & 3B & 2B & 1B & BB \\
	\hline
	Min & 0.83 & 0.66 & 0.87 & 0.92 & 0.94 \\
	\hline
	Max & 1.24 & 1.69 & 1.23 & 1.08 & 1.08 \\
	\hline
     \end{tabular}
\end{center}

In order to assign these values to a park, for further balancing purposes, parks will start at a flat 1.00 multiplier and
 will adjust towards the max/min values via a “Point Buy” system. Teams will have 15 points to spend on adjustments
, and each point will be worth 1/10th of the distance to maximum or minimum values in each factor, with a limit of 10
 points per factor. For example, adjusting HR factors would be +0.024 or -0.017 per point.

\begin{center}
      \begin{tabular}{|c|c|c|c|c|c|}
	\hline
	Effect per point & HR & 3B & 2B & 1B & BB \\
	\hline
	Towards Max & 0.024 & 0.069 & 0.023 & 0.008 & 0.006 \\
	\hline
	Towards Min & 0.017 & 0.034 & 0.013 & 0.008 & 0.008 \\
	\hline
     \end{tabular}
\end{center}
Park factors are maintained on the rosters sheet.