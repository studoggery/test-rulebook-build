%! TEX root = ../rulebook.TEX

\subsection{Game mechanics}
\begin{deepEnumerate}
	\item Standard plate appearances
	\begin{deepEnumerate}
		\item Each plate appearance involves a pitch number and a swing number, each between 1 and 1000 inclusive.
		\item The 1-1000 range wraps around, meaning a pitch number of 10 is 10 away from both hitter numbers 20 and 1000.
		\item The difference between these two numbers, combined with the types and hands of the pitcher and batter, 
		determines the outcome of the plate appearance. 
		See \nameref{sec:ranges index} for a deeper explanation of this mechanic 
		and for a listing of batter and pitcher types and hand bonuses.
		\item Players can be right-handed or left-handed. 
		A pitcher with the same handedness as the batter they are facing receives a slight bonus to their numbers, 
		increasing the likelihood of recording an out. 
		This is determined by the pitcher's hand bonus (see \nameref{sec:ranges index}).
		\item Teams may also elect to utilize custom park factors for all their home games. 
		These factors apply to both teams playing in the game. 
		See \nameref{sec:ranges index} for an explanation of how they work.
		\begin{deepEnumerate}
			\item Park factors affect every game played at a team's home stadium
			\item Park factors must be submitted during the offseason and will be used for the following season. 
			\item Park factors cannot be changed midseason.
		\end{deepEnumerate}
		\item The following table describes each possible outcome of a plate appearance.
		\begin{center}
			\begin{longtable}{|p{3cm}|p{8cm}|}
				\hline
				\textbf{Outcome}           & \textbf{Description}                                                  \\
				\hline
				Home run (HR)              & Everyone on base as well as the batter scores.                        \\
				\hline
				Triple (3B)                & Everyone on base scores.                                              
				The batter is placed on third base. \\
				\hline
				Double (2B)                & Anyone on third base or second base scores.                           
				Anyone on first base moves to third base.
				The batter is placed on second base. \\
				\hline
				Single (1B)                & Anyone on third base scores.                                          
				Anyone on second base moves to third base.
				Anyone on first base moves to second base.
				The batter is placed on first base. \\
				\hline
				Walk (BB)                  & Anyone on third base scores if there are runners on first and second. 
				Anyone on second base moves to third base if there is a runner on first base.
				Anyone on first base moves to second base.
				The batter is placed on first base. \\
				\hline 
				Flyout (FO)                & The batter is out.                                                    
				Anyone on third base with less than two outs scores. 
				This is scored as a sacrifice fly and does not count as an at-bat. \\
				\hline
				Strikeout (K)              & The batter is out.                                                    \\
				\hline
				Popup (PO)                 & The batter is out.                                                    \\
				\hline
				Right-side groundout (RGO) & The batter is out.                                                    
				If there is a runner on first base with fewer than two outs, the runner and batter are both out as a double play.
				On any right-side groundout that does not end the inning:
				Anyone on third base scores.
				Anyone on second base moves to third base. \\
				\hline
				Left-side groundout (LGO)  & The batter is out.                                                    
				If there is a runner on first base with fewer than two outs, the runner and batter are both out as a double play.
				On any left-side groundout that does not end the inning:
				Anyone on third base scores.
				Anyone on second base stays at second base, unless there is a runner at first. \\
				\hline
				Triple Play (TP) & If there are zero outs and runners on at least first base and second base,                                                    
				a plate appearance resulting in a difference of between 496 and 500 inclusive results in a triple play. 
				Three outs are recorded and no runs score. \\
				\hline
			\end{longtable}
		\end{center}
		\item When there are two outs, any hit advances runners one extra base than it normally would. 
		This does not apply to the batter.
		\begin{deepEnumerate}
			\item For example, if a batter hits a single with two outs and runners at first and second, 
			the runner on second scores, the runner on first moves to third, and the batter is placed on first base.
		\end{deepEnumerate}
	\end{deepEnumerate}
	\item Steals
	\begin{deepEnumerate}
		\item A runner may attempt to steal any time they are on base and the base ahead of them is unoccupied.
		\begin{deepEnumerate}
			\item The steal may be initiated by the GM or the player. Steals are initiated by submitting a steal number to the umpire.
			\item The steal must be submitted before the next batter swings.
			\begin{deepEnumerate}
				\item The auto timer for the current batter's plate appearance pauses when a steal is submitted and resumes when the batter is pinged on reddit with a new pitch.
			\end{deepEnumerate}
			\item Multi-runner steals are allowed. They must be explicitly mentioned by the player or GM initiating the steal.
			\begin{deepEnumerate}
				\item Only the lead runner is considered for the purpose of determining success or failure. Trailing runners will always succeed.
			\end{deepEnumerate}
			\item Pitchers have the option of submitting a second number explicitly marked as a steal pitch before an at-bat. 
			If a steal is attempted during the at-bat, the pitcher's steal number is used in the outcome instead of the original pitch. 
		\end{deepEnumerate}
		\item Outcomes of steals are determined according to the following tables.
		\begin{center}
			\begin{tabular}{|c|c|c|c|}
				\hline
				\textbf{Tier} & \textbf{Stealing 2nd} (SB2)    & \textbf{Stealing 3rd} (SB3)    & \textbf{Stealing home} (SB4) \\
				\hline 
				Tier 1   	  & 0-335                          & 0-185                          & 0-35                         \\
				\hline
				Tier 2    	  & 0-320                          & 0-165                          & 0-30                         \\
				\hline
				Tier 3   	  & 0-300                          & 0-150                          & 0-25                         \\
				\hline
				Tier 4   	  & 0-280                          & 0-135                          & 0-20                         \\
				\hline
				Tier 5   	  & 0-265                          & 0-115                          & 0-15                         \\
				\hline
			\end{tabular}
			\begin{tabular}{|c|c|}
				\hline
				\textbf{Tier} & \textbf{Type}                                           		\\
				\hline 
				Tier 1   	 & Speedy \\
				\hline
				Tier 2    	 & Extra Base Focused, Sacrifice Master\\
				\hline
				Tier 3   	 &Basic Contact, Extreme Neutral, Basic Neutral, Basic Power, Single Focused, Work the Count \\
				\hline
				Tier 4   	 & 1B/BB, Three True Outcomes, HR/K \\
				\hline
				Tier 5   	 & Max Homers, Pitcher \\
				\hline
			\end{tabular}
			
		\end{center}
	\end{deepEnumerate}
	\item Bunts
	\begin{deepEnumerate}
		\item A batter may elect to bunt. This must be clearly stated when submitting a swing number.
		\item The outcome of a bunt is determined according to the following table.        
		\begin{center}
			\begin{tabular}{|p{3cm}|p{8cm}|}
				\hline
				\textbf{Difference} & \textbf{Outcome}                                 \\
				\hline 
				0-50 inclusive                 & The play is treated the same as a normal single. \\
				\hline 
				51-375 inclusive               & The batter is out on a sacrifice.                
				Anyone on third base does not advance.
				Anyone on second advances if third base is not occupied.
				Anyone on first advances if second or third base is unoccupied. \\
				\hline 
				376-475 inclusive              & The batter strikes out.                          \\
				\hline
				476-500 inclusive              & The batter is out.                               
				Anyone on first base is out in a double play.
				If there are runners on first and second and there are zero outs,
				the runner on second advances to third and the batter and runner on first are out in a double play. \\
				\hline
			\end{tabular}
		\end{center}
    \end{deepEnumerate}
    \item Infield in
    \begin{deepEnumerate}
        \item With a runner on third base, the pitcher, GM, or team captains of the team currently in the field 
		can ask for the infield in at any time prior to the next swing.
		\begin{deepEnumerate}
			\item If the pitcher, GM, or team captain differ in deciding to bring the infield in, 
        	the umpire shall give priority in the following order: GM, Primary Captain, Pitcher, Secondary Captain.
			\begin{deepEnumerate}
            	\item In the event two secondary captains differ, infield in will not be called.
			\end{deepEnumerate}
		\end{deepEnumerate}
        \item When the infield is in, a modifier of +18 to singles, -9 to RGO, and -9 to LGO 
        will be added to the resulting ranges for the at bat to be applied after park factors.
        \item When the bases are loaded, an RGO or LGO results in an out at the plate and all other runners including the batter advancing.
        \item When the bases are not loaded, an RGO or LGO results in the runner being held at third base, and the batter is out at first. 
        A runner at first base will advance to second.
        \item There are no double or triple plays when the infield is in. 
        \item Ranges for bunting and stealing are not impacted by the infield being in.
    \end{deepEnumerate}
    \item IBB (Intentional Base on Balls)
    \begin{deepEnumerate}
        \item The current batter is issued an automatic walk. 
        \item An IBB can be submitted to the ump by either a pitcher or the GM. 
        \item The pitcher timer does not reset after an IBB.
    \end{deepEnumerate}
\end{deepEnumerate}
