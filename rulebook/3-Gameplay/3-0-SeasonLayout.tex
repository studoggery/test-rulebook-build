%! TEX root = ../rulebook.tex

\subsection{Season Layout}
\begin{deepEnumerate}
	\item Seasons consist of a schedule of 16 games followed by playoffs.
	\item Games are played one at a time, in periods of time called "sessions".
	\begin{deepEnumerate}
		\item Sessions are 10 days long.
		\item Sessions begin two days after the previous session ends.
		\begin{deepEnumerate}
			\item Exceptions to this may occur due to holidays or shortage of league officials.
		\end{deepEnumerate}
		\item Games begin at 12PM ET on the first day of the session.
		\item Games that are not completed by 12pm ET on the final day of the session
		(240 hours after the start of the session) are designated as shortened games.
		\begin{deepEnumerate}
			\item In regular season games, if the home team is winning and the game is in the bottom of the inning
			when the session concludes,	the game will automatically end with the home team declared the winner.
			\item Otherwise, in regular season games, games will end at the end of the current inning.
			If the game is tied, the inning is treated as the 6th inning, and any further innings will be played under extra inning baserunner rules
			(see \hyperref[sec:extra innings]{Extra innings}).
			\item All games that run over the end of the session are subject to shortened extra-inning auto timers
			(see \hyperref[sec:extra innings]{Extra innings}) effective immediately upon session end.
		\end{deepEnumerate}
		\item Playoff format
		\begin{deepEnumerate}
			\item From each league, three division champions and three wild card teams will advance to the playoffs.
			\item The playoffs consist of four rounds:
			the Wild Card round, the Divisional Games, the League Championship Games, and the Paper Cup.
			Teams will be seeded first by division winners followed by the wild card teams.
			\begin{deepEnumerate}
				\item The Wild Card round is the first round of the playoffs. 
				In each league the wild card teams and the lowest seeded division winner will face off. 
				The division winner will host the lowest seeded wild card team 
				and the highest seeded wild card team will host the middle seeded wild card team.
				\item The Divisional Games is the league semifinal round. 
				In each league, the top seeded team hosts the lower ranked winning team from the Wild Card round, 
				and the second seeded team hosts the higher seeded winning team from the Wild Card round.
				\item The League Championship Game is the game to decide the American League and National League champions. 
				The higher of the two remaining seeds hosts the lower seed.
				\item The Paper Cup is the game that decides the champion of the Major League Redditball season. 
				Home field advantage is determined by regular season record, 
				with the wild card tiebreakers applied as necessary.
			\end{deepEnumerate}
			\item Tiebreakers
			\begin{deepEnumerate}
				\item If at the end of the regular season two or more clubs finish with identical winning percentages, the following steps
				will be taken to determine their final order of finish.
				\begin{deepEnumerate}
					\label{sec:Division tiebreakers}
					\item Division tiebreakers:
					1) Head-To-Head Record $\rightarrow$  
					2) Record vs Division $\rightarrow$
					3) Best win-loss percentage in common games $\rightarrow$ 
					4) Strength of Victory $\rightarrow$ 
					5) Strength of Schedule $\rightarrow$ 
					6) Run Differential $\rightarrow$  
					7) Number of Autos $\rightarrow$ 
					8) Home Run Derby (10 outs)
					\label{sec:Wildcard tiebreakers}
					\item Two Team Wildcard tiebreakers:
					1) Division tiebreakers if all tied teams are in the same division $\rightarrow$ 
					2) Head-To-Head Record $\rightarrow$  
					3) Record vs Common Opponents (minimum of four games) $\rightarrow$ 
					4) Strength of Victory $\rightarrow$ 
					5) Strength of Schedule $\rightarrow$ 
					6) Run Differential $\rightarrow$  
					7) Number of Autos $\rightarrow$  
					8) Home Run Derby (10 outs)
					\item Three+ Team Wildcard tiebreakers:
					\begin{deepEnumerate}
						\item Apply Division tiebreakers to eliminate all but the highest ranked team in each division.
						The original seeding within a division after applying the Division tiebreaker remains the same for all
						subsequent applications of tiebreaker procedures. Then:
						\item 1) Head-To-Head Sweep (one team won or lost all games against all others)  $\rightarrow$  
						2) Record vs Common Opponents (minimum of four games) $\rightarrow$ 
						3) Strength of Victory $\rightarrow$ 
						4) Strength of Schedule $\rightarrow$ 
						5) Run Differential $\rightarrow$  
						6) Number of Autos $\rightarrow$  
						7) Home Run Derby (10 outs)
						\item When the first Wild-Card team has been determined, the full procedure is repeated among remaining tied teams 
						to determine the second Wild-Card team.
					\end{deepEnumerate}
				\end{deepEnumerate}
				\item In the event of multiple teams being tied for both a division and a wildcard spot, the division tiebreaker will be run first.
				Once all division ties have been broken, the wildcard tiebreaker will then be ran.
				\item Only one club advances to the playoffs in any tie-breaking step. If two clubs remain tied in any tie-breaker step
				after all other clubs have been eliminated, the procedure reverts to Step 1 of the two-club format to determine the winner.
				When one club wins the tiebreaker, all other clubs revert to Step 1 of the applicable two-club or three-club format.
				\item In comparing records against common opponents among tied teams, the best win-loss percentage is the deciding factor,
				as teams may have played an unequal number of games.
			\end{deepEnumerate}
		\end{deepEnumerate}
	\end{deepEnumerate}
\end{deepEnumerate}
